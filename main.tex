
%%%%%%%%%%%%%%%%%%%%%%%%%%%%%%%%%%%%%%%%%%%%%%%%%%%%%%%%%%%%%%%%%%%%%%
% UNIVERSIDADE FEDERAL DO PARÁ
% INSTITUTO DE CIÊNCIAS EXATAS E NATURAIS
% FACULDADE DE COMPUTAÇÃO
% DISCIPLINA: GRAFOS
% DOCENTE: ROBERTO SAMARONE
% DISCENTE: MÁRIO DIEGO VALENTE
%%%%%%%%%%%%%%%%%%%%%%%%%%%%%%%%%%%%%%%%%%%%%%%%%%%%%%%%%%%%%%%%%%%%%%

\documentclass[12pt]{article}
\usepackage{sbc-template}
\usepackage{graphicx,url}
\usepackage{xcolor} 
\usepackage[utf8]{inputenc}
\usepackage[portuguese]{babel}
\usepackage{quoting}
\usepackage{hyphenat}
\usepackage{hyperref}
\usepackage{multicol}
\usepackage{tikz}
\usetikzlibrary{arrows.meta,positioning} 
\usepackage{amssymb}
\usepackage{algorithm}
\usepackage{algpseudocode}


\sloppy
\title{O Problema de Fluxo Máximo em Redes: Implementação do Algoritmo de Ford-Fulkerson via Python}

\author{Mário Diego Rocha Valente\inst{1}, Fabrício dos Santos Assunção\inst{1},Roberto Samarone dos Santos\inst{2}}

\address{Graduando em Sistemas de Informação - Universidade Federal do Pará(UFPA) \\
  Rua. Augusto Corrêa 01 -- Guamá -- Belém -- PA -- Brasil
\nextinstitute
  Professor Dr., Faculdade de Computação - Universidade Federal do Pará(UFPA)\\
  Av. Augusto Corrêa 01 -- Guamá -- Belém - -- PA -- Brasil \\
  \email{diego.vatente@gmail.com, fabricioassuncao71@gmail.com, rsa@ufpa.br}
  }



\begin{document} 

\maketitle

\begin{abstract}
  This paper describes.....
\end{abstract}
     
\begin{resumo} 
  Este artigo apresenta um estudo consolidado sobre o problema de fluxo em redes, com foco no algoritmo de Ford–Fulkerson. Partimos de uma contextualização histórica e prática do tema, passando por definições formais, pelo Teorema do Fluxo Máximo–Corte Mínimo, pelos problemas clássicos modeláveis via fluxo, até a descrição detalhada do algoritmo e sua implementação passo a passo em um exemplo ilustrativo. A contribuição principal é reunir, em formato didático e alinhado a requisitos acadêmicos, os conceitos essenciais e uma execução comentada que pode servir de base para atividades práticas na disciplina de Teoria dos Grafos.
\end{resumo}

\newpage
\section{Introdução}

A teoria dos grafos consolidou-se como um dos pilares matemáticos e computacionais para a modelagem de problemas de conectividade, transporte e alocação de recursos. Desde o problema das pontes de Königsberg, proposto por Euler no século XVIII \cite{euler1736}, a ideia de representar entidades como vértices e relações como arestas revelou-se uma abstração poderosa, cuja importância se ampliou ao longo do século XX em áreas como Pesquisa Operacional, Ciência da Computação e Engenharia \cite{bondy1976,west2001}.

\noindent Entre as diversas classes de problemas em grafos, destacam-se os \emph{problemas de fluxo em redes}, que modelam o escoamento de recursos de uma fonte $s$ até um sorvedouro $t$, respeitando restrições de capacidade nos arcos. Essa formulação permite descrever fenômenos de natureza diversa, como transporte de mercadorias, tráfego rodoviário, transmissão de dados, distribuição de energia elétrica e abastecimento de água \cite{ahuja1993,ahuja1990}. Em todos esses contextos, a questão central é determinar qual a taxa máxima de transferência possível sem violar as limitações impostas pela rede.

\noindent Um marco fundamental nesse campo foi a proposta do método de Ford–Fulkerson, publicada em 1956 \cite{ford1956}. Essa abordagem introduziu o conceito de \emph{caminhos aumentantes} no \emph{grafo residual} para, de forma iterativa, ampliar o valor do fluxo até alcançar sua máxima capacidade. O método está intimamente relacionado ao \emph{Teorema do Fluxo Máximo–Corte Mínimo}, formalizado por Ford e Fulkerson \cite{ford1956}, que estabelece que o valor máximo de fluxo em uma rede é exatamente igual à capacidade mínima de um corte que separa a fonte do sorvedouro. Essa equivalência teórica não apenas fundamenta o método de Ford–Fulkerson, mas também abriu caminho para uma vasta família de algoritmos posteriores \cite{cormen2009}.

\noindent A partir dessa formulação original, surgiram variações que buscam maior eficiência computacional. O algoritmo de Edmonds–Karp (1972) \cite{edmonds1972} define o uso sistemático da busca em largura (BFS), garantindo complexidade polinomial. O algoritmo de Dinic (1970) \cite{dinic1970} introduziu o conceito de grafos de nível e fluxos bloqueadores, ampliando o desempenho em cenários específicos. Posteriormente, técnicas como \emph{push–relabel} (Goldberg–Tarjan, 1986) \cite{goldberg1988} exploraram estratégias locais de redistribuição de fluxo, permitindo lidar com redes de grande escala. Essas contribuições consolidaram o problema de fluxo máximo como uma das áreas mais férteis da otimização combinatória.

\noindent No contexto contemporâneo, algoritmos de fluxo são aplicados não apenas em sistemas físicos e de transporte, mas também em domínios avançados como segmentação de imagens \cite{boykov2001}, análise de dados em grafos \cite{ahuja1993}, \emph{routing} em redes de computadores \cite{kleinberg2006} e planejamento de evacuação em situações de risco \cite{chen2010}. Essa versatilidade evidencia a relevância prática do problema e a necessidade de compreender suas bases teóricas e implementações.

\noindent Este artigo tem como objetivo apresentar uma análise do problema de fluxo máximo e algoritmo de Ford–Fulkerson. Serão discutidos seus fundamentos conceituais, pseudocódigo e variações, limitações e um estudo de caso com implementação prática, a fim de evidenciar seu papel como ferramenta de modelagem e otimização.


\section{Trabalhos Correlatos} \label{sec:firstpage}

A teoria de fluxo em redes e o algoritmo de Ford-Fulkerson constituem pilares fundamentais na pesquisa operacional e na ciência da computação, sendo amplamente estudados desde a década de 1950. A seguir, apresenta-se uma visão histórica e evolutiva dos principais trabalhos correlatos, suas contribuições e variações do algoritmo.

\noindent O algoritmo de Ford-Fulkerson foi proposto por \textbf{L.R. Ford Jr. e D.R. Fulkerson} em 1956, no artigo seminal \cite{ford1956}. O método introduziu conceitos fundamentais, como rede residual, caminho aumentante e corte mínimo, consolidando o problema do fluxo máximo como uma área central na otimização de redes e em problemas como emparelhamento em grafos bipartidos.

Em 1972, \textbf{Jack Edmonds e Richard Karp} apresentaram uma melhoria significativa, escolhendo sempre o caminho aumentante mais curto (em número de arestas) por meio de uma busca em largura (BFS) \cite{edmonds1972}. Essa implementação, conhecida como \textit{Algoritmo de Edmonds-Karp}, garante convergência em tempo polinomial, com complexidade $O(V \cdot E^2)$, evitando ciclos infinitos que poderiam ocorrer no algoritmo original.

Na década de 1970, \textbf{Yefim Dinic} desenvolveu o conceito de grafo de nível e fluxos bloqueadores \cite{dinic1970}, permitindo encontrar múltiplos caminhos aumentantes simultaneamente. O algoritmo de \textit{Dinic} tem complexidade $O(V^2 \cdot E)$ no caso geral e $O(\min(V^{2/3}, E^{1/2}) \cdot E)$ em grafos unitários, representando um avanço no desempenho do algoritmo original.

Em 1986, \textbf{Goldberg e Tarjan} propuseram o método \textit{Push-Relabel} \cite{goldberg1988}, baseado em pré-fluxos, que violam temporariamente a conservação do fluxo e são ajustados localmente por operações de \textit{push} e \textit{relabel}. Essa abordagem permite redistribuir fluxo sem depender de caminhos aumentantes globais, alcançando complexidade $O(V^3)$ no caso geral, com refinamentos que reduzem para $O(V^2 \sqrt{E})$.

\textbf{Ahuja e Orlin} (1992) propuseram o \textit{Capacity Scaling} \cite{ahuja1993}, otimizando Ford-Fulkerson ao considerar apenas caminhos com capacidade acima de um limite, reduzindo buscas inúteis. Paralelamente, surgiram algoritmos distribuídos e paralelos, voltados a redes de computadores e processamento de grafos massivos, consolidando a aplicabilidade prática da teoria de fluxo em larga escala.

\textbf{James Orlin} (2013) anunciou o algoritmo de fluxo máximo mais rápido conhecido até então \cite{orlin2013maxflow}, com complexidade $O(VE)$. \textbf{Goldberg e Rao} desenvolveram o algoritmo assintoticamente mais rápido para fluxo máximo, baseado em fluxos de bloqueio \cite{goldberg1998}, com tempo de execução $O\big(\min(V^{2/3}, E^{1/2}) \cdot E \cdot \log(V^2/E + 2) \cdot \log C \big)$, sem utilizar push-relabel.



\newpage
\section{Redes de Fluxo  (ou Rede Capacitada)} 

As redes de fluxo constituem um modelo matemático fundamental em teoria dos grafos, aplicado na resolução de problemas de transporte, logística e comunicação \cite{ahuja1993,cormen2009}.

Uma rede de fluxo é um grafo direcionado, $G = (V, E)$, onde:

\begin{itemize}
    \item $V$ é o conjunto de vértices (nós) da rede.
    \item $E$ é o conjunto de arestas (arcos) direcionadas que conectam os vértices.
\end{itemize}

\noindent A cada aresta $(u, v) \in E$ está associado um número real não negativo $c(u, v)$, denominado capacidade da aresta. A capacidade $c(u, v)$ representa a quantidade máxima de fluxo que pode passar de $u$ para $v$ por unidade de tempo. Se não houver aresta de $u$ para $v$, convencionalmente $c(u, v) = 0$.

\noindent Existem dois vértices especiais: uma fonte (s), que é o ponto de origem do fluxo, e um sumidouro (t), que é o ponto de destino ou consumo do fluxo. Todos os outros vértices são intermediários e apenas propagam o fluxo.

\subsection{Fluxo} 

Um fluxo em uma rede de fluxo $G$ é uma função $f: V \times V \rightarrow \mathbb{R}$ (que associa um número real a cada par de vértices) que satisfaz as seguintes propriedades \cite{ahuja1993},:

\begin{itemize}
    \item \textbf{Restrição de Capacidade}: Para cada par de vértices $(u, v) \in V \times V$, o fluxo $f(u, v)$ deve ser não negativo e não pode exceder a capacidade da aresta: $0 \le f(u, v) \le c(u, v)$ Se não houver aresta $(u, v)$ no grafo original, então $f(u, v) = 0$;
    \item Conservação do Fluxo: Para cada vértice $u \in V$, exceto a fonte $s$ e o sumidouro $t$, o fluxo total que entra em $u$ deve ser igual ao fluxo total que sai de $u$: $\sum_{v \in V} f(v, u) = \sum_{v \in V} f(u, v)$ para todo $u \in V \setminus {s, t}$ Isso significa que não há acúmulo nem perda de material nos vértices intermediários;
    \item \textbf{Valor do Fluxo}: O valor de um fluxo $f$, denotado por $|f|$, é o fluxo líquido total que sai da fonte $s$: $|f| = \sum_{v \in V} f(s, v) - \sum_{v \in V} f(v, s)$ Normalmente, em uma rede de fluxo, não há arestas de entrada na fonte, então o segundo termo é zero. O objetivo do problema de fluxo máximo é encontrar um fluxo $f$ tal que seu valor $|f|$ seja o maior possível;
    \item \textbf{Redes com Múltiplas Fontes e Sumidouros}: O problema de fluxo máximo pode ser estendido para redes com múltiplas fontes e múltiplos sumidouros \cite{goldberg1988,ahuja1993}.Para resolver esses casos, a rede original pode ser transformada em uma rede equivalente com uma única fonte e um único sumidouro:
    \begin{itemize}
        \item Superfonte (s’): É adicionada uma nova superfonte $s'$ e arestas direcionadas de $s'$ para cada fonte original $s_i$ com capacidade infinita $c(s', s_i) = \infty$;
        \item Supersumidouro (t’): É adicionado um novo supersumidouro $t'$ e arestas direcionadas de cada sumidouro original $t_j$ para $t'$ com capacidade infinita $c(t_j, t') = \infty$. Qualquer fluxo na rede original corresponde a um fluxo na rede transformada, e vice-versa.
    \end{itemize}
    \item \textbf{ Capacidades em Vértices}: Em algumas redes, os vértices também podem ter capacidades que limitam a quantidade de fluxo que pode passar por eles. Esse problema também pode ser transformado em uma rede equivalente onde todas as capacidades estão nas arestas. Isso é feito dividindo cada vértice $v$ com capacidade $l(v)$ em dois vértices, $v_{in}$ e $v_{out}$, conectados por uma aresta $(v_{in}, v_{out})$ com capacidade $c(v_{in}, v_{out}) = l(v)$. Todas as arestas de entrada para $v$ agora vão para $v_{in}$, e todas as arestas de saída de $v$ agora saem de $v_{out}$.
\end{itemize}
 
\noindent O problema de fluxo máximo, portanto, busca o valor máximo de fluxo possível, respeitando todas as capacidades das arestas e as leis de conservação de fluxo, tornando-o uma ferramenta fundamental para otimizar a passagem de recursos através de qualquer sistema que possa ser modelado como uma rede.

\section{Teorema do Fluxo Máximo/Corte Mínimo}

Um dos resultados mais importantes da teoria de redes de fluxo é o Teorema do Fluxo Máximo, formulado por Ford e Fulkerson \cite{ford1956maximal}.

\noindent Para entender este teorema, precisamos primeiro definir o conceito de um corte em uma rede de fluxo. Um corte (S, T) em uma rede de fluxo $G=(V, E)$ com fonte $s$ e sumidouro $t$ é uma partição do conjunto de vértices $V$ em dois subconjuntos $S$ e $T = V \setminus S$, tal que a fonte $s \in S$ e o sumidouro $t \in T$.

A capacidade de um corte $c(S, T)$ é a soma das capacidades de todas as arestas que vão de um vértice em $S$ para um vértice em $T$: $c(S, T) = \sum_{(u,v) \in E \text{ com } u \in S, v \in T} c(u, v)$ Note que arestas de $T$ para $S$ não contribuem para a capacidade do corte neste sentido, pois o fluxo é unidirecional de $S$ para $T$.

\noindent O Teorema do Fluxo Máximo/Corte Mínimo \cite{cormen2009}, estabelece que o valor do fluxo máximo em uma rede é igual à capacidade de um corte mínimo na mesma rede. Um corte mínimo é um corte cuja capacidade é a menor entre todos os cortes possíveis na rede.

\noindent Este teorema é central porque fornece uma condição de otimalidade para o fluxo encontrado. Seja $f$ um fluxo em uma rede $G$ com fonte $s$ e sumidouro $t$. As seguintes condições são equivalentes:


\begin{enumerate}
    \item $f$ é um fluxo máximo em $G$.
    \item  A rede residual $G_f$ não contém nenhum caminho aumentante.
    \item $|f| = c(S, T)$ para algum corte $(S, T)$ de $G$.
\end{enumerate}

\noindent A equivalência dessas condições é crucial: se um algoritmo encontra um fluxo $f$ e não consegue mais encontrar caminhos aumentantes na rede residual $G_f$, então ele encontrou um fluxo máximo. Além disso, a capacidade desse fluxo máximo será exatamente igual à capacidade do corte mínimo da rede, que é o "gargalo" ou a "vulnerabilidade" da rede que limita o fluxo total. 

\noindent A prova formal deste teorema é um processo de indução que demonstra que, para qualquer fluxo, o valor do fluxo é sempre menor ou igual à capacidade de qualquer corte, e que para o fluxo máximo existe um corte cuja capacidade é exatamente igual ao valor do fluxo.

\newpage
\subsection{Algoritmo de Ford–Fulkerson (pseudocódigo)} 

A seguir apresentamos um pseudocódigo em estilo compatível com a literatura clássica (CLRS), porém redigido neste trabalho.\footnote{Cormen et al., (2009).} A estratégia geral é iterar enquanto existir caminho aumentante entre $s$ e $t$ no grafo residua

\begin{algorithm}[H]
\caption{Algoritmo de Ford–Fulkerson}
\label{alg:ford_fulkerson}
\begin{algorithmic}[1]
\Procedure{FordFulkerson}{$G=(V,E), s, t$}
    \State Inicialize $f(u,v) \gets 0$ para toda aresta $(u,v) \in E$
    \While {existe caminho aumentante $P$ de $s$ até $t$ na rede residual $G_f$}
        \State Construir o grafo residual $G_f$ a partir do fluxo $f$
        \State Encontrar um caminho aumentante $P$ de $s$ até $t$ em $G_f$ (ex.: DFS ou BFS)
        \State $\Delta \gets \min\{c_f(u,v) \mid (u,v) \in P\}$ 
              \Comment{Capacidade residual mínima de $P$}
        \For {cada aresta $(u,v)$ em $P$}
            \State $f(u,v) \gets f(u,v) + \Delta$ 
                   \Comment{Aumenta fluxo na direção direta}
            \State $f(v,u) \gets f(v,u) - \Delta$ 
                   \Comment{Permite fluxo reverso (desfazer se necessário)}
        \EndFor
    \EndWhile
    \State \Return $f$ \Comment{Fluxo máximo encontrado}
\EndProcedure
\end{algorithmic}
\end{algorithm}



\noindent Se a busca por caminhos aumentantes for feita por BFS (sempre escolhendo o caminho mais curto em número de arestas), obtém-se a variante \emph{Edmonds–Karp}, cuja complexidade é $\mathcal{O}(|V|,|E|^2)$. Com uma estratégia arbitrária (DFS), o número de iterações pode depender do valor do fluxo máximo quando as capacidades são inteiras.


\subsection{Problemas clássicos Modeláveis como Fluxo} 

Além do próprio fluxo máximo e do corte mínimo, formulam-se como problemas de fluxo: 

\begin{itemize} 
\item \textbf{Casamento bipartido máximo:} construir rede $s\to$ lado esquerdo, arestas de capacidade 1 entre as partições, e lado direito $\to t$. 
\item \textbf{Atribuição (assignment) e escalonamento:} variantes com restrições de capacidade e custos (em min-cost flow). 
\item \textbf{Circulação com demandas:} generaliza fluxo, permitindo balanços em vértices e cotas em arestas; resolve cadeias de suprimento. 
\item \textbf{Cortes mínimos e \emph{s-t} cut:} computáveis a partir do grafo residual final; aplicam-se a segmentação de imagens e vis~ao computacional. 
\item \textbf{Caminhos disjuntos e conectividade:} via reduções que impõem capacidades unitárias. \end{itemize}



\subsection{Aplicações Práticas}

O problema de fluxo máximo e suas variações têm ampla aplicação em diversas áreas:


\begin{itemize}
    \item \textbf{Tráfego Urbano}: otimização de fluxos de veículos, identificação de congestionamentos e ajustes de direção de ruas \cite{saidane2002traffic, gupta2016traffic, dolgopolov2019traffic, gong2022traffic, iemini1994dynamic};
    \item \textbf{Transporte e Logística}: maximização de rotas, distribuição de produtos e escalas de voo;
    \item \textbf{Redes de Energia e Água}: otimização na transmissão de eletricidade e distribuição de água \cite{bulut2021optimization};
    \item \textbf{Redes de Comunicação}: maximização da largura de banda, codificação em redes ad-hoc \cite{neto2015communication, zhang2005ad};
    \item \textbf{Visão Computacional}: segmentação de imagens e minimização de energia \cite{boykov2001};
    \item \textbf{Balanceamento de Carga}: roteamento dinâmico e balanceamento em redes computacionais \cite{tsiaka2008load, peijun2011routing}.
\end{itemize}


Essa trajetória histórica demonstra como o algoritmo de Ford-Fulkerson evoluiu e se adaptou, consolidando-se como um pilar teórico e prático na ciência da computação e em aplicações de engenharia e logística.


\section{Vantagens e Desvantagens} \label{sec:vantagens}

O algoritmo de Ford–Fulkerson apresenta um papel fundamental na resolução de problemas de fluxo em redes, com aplicações que vão desde sistemas de transporte até otimização industrial. No entanto, como todo método computacional, ele possui vantagens e limitações que precisam ser consideradas de acordo com o contexto de uso.


\subsection{Vantagens}

Uma das principais vantagens do algoritmo é a sua \textbf{simplicidade conceitual}. A ideia de encontrar caminhos aumentantes sucessivos até que não seja mais possível ampliar o fluxo torna o método intuitivo e relativamente fácil de implementar, tanto em termos matemáticos quanto computacionais.

Outro ponto positivo é a \textbf{versatilidade}. O Ford–Fulkerson pode ser aplicado em diversos cenários reais, como:


\begin{itemize} 
    \item \textbf{Fluxo de trânsito}: ao modelar ruas e avenidas como arestas com capacidades limitadas (representando o número de veículos que podem passar por unidade de tempo), é possível identificar gargalos no tráfego e otimizar a circulação de veículos em grandes centros urbanos.      
    \item \textbf{Indústria metalúrgica}: no problema do corte de chapas metálicas, o algoritmo auxilia na determinação de rotas ótimas de produção, maximizando o uso do material e minimizando desperdícios. 
    \item \textbf{Fluxo em tubulações}: redes de abastecimento de água, petróleo ou gás podem ser modeladas como grafos, onde as capacidades representam o volume máximo de fluido. O algoritmo auxilia na previsão de gargalos e no planejamento de expansões da rede. 
\end{itemize}

Além disso, uma vantagem teórica é a \textbf{garantia de encontrar a solução ótima}, desde que as capacidades sejam inteiras, o que é extremamente útil em problemas práticos que envolvem unidades discretas de recursos.






\subsection{Desvantagens}

Entre as desvantagens, destaca-se a \textbf{dependência da escolha dos caminhos aumentantes}. O tempo de execução do algoritmo pode variar significativamente conforme a ordem em que os caminhos são encontrados. Em certos casos, essa dependência pode levar a tempos de execução muito longos, especialmente quando as capacidades não são inteiras.

Outra limitação é a \textbf{eficiência computacional}. Embora seja eficiente para grafos de tamanho moderado, em aplicações de grande escala — como redes de telecomunicações globais ou sistemas logísticos complexos — o Ford–Fulkerson pode não ser a escolha mais adequada, sendo preferível o uso de algoritmos mais eficientes como Edmonds–Karp ou Push–Relabel.

Por fim, uma desvantagem prática é que o algoritmo \textbf{não lida diretamente com restrições adicionais}, como custos associados ao fluxo. Em aplicações reais, como logística de transporte ou redes de energia, além do fluxo máximo, é necessário considerar custos e eficiência, o que exige adaptações ou o uso de variantes do algoritmo.





\section{Resultados}
\subsection{Estudo de Caso} \label{sec:implementacao} 

Nesta seção, ilustramos o funcionamento do Ford–Fulkerson em uma rede clássica. Usaremos os vértices ${S, A, B, C, D, E, F, G, H, I, J, K, L , M, T}$ e as capacidades mostradas na Figura,\ref{fig:grafo}. O objetivo é determinar o fluxo máximo de $S$ para $T$.\vskip1.5cm





\begin{tikzpicture}[>=Stealth, font=\small]
  % estilo dos nós
  \tikzset{v/.style={circle, draw, minimum size=8mm, inner sep=1pt, align=center}}

  % --- Coordenadas fixas ajustadas ---
  \node[v] (S) at (0,0) {S};

  % Linha superior
  \node[v] (A) at (2,2.2)  {A};
  \node[v] (C) at (4,2.2)  {C};
  \node[v] (F) at (6,2.2)  {F};
  \node[v] (J) at (8,2.2)  {J};
  \node[v] (L) at (10,2.2) {L};

  % Linha central
  \node[v] (E) at (2,0.2)  {E};
  \node[v] (G) at (6,0.2)  {G};
  \node[v] (H) at (8,-0.2)  {H};

  % Linha inferior
  \node[v] (B) at (2,-2.2) {B};
  \node[v] (D) at (4,-2.2) {D};
  \node[v] (I) at (6,-2.2) {I};
  \node[v] (K) at (8,-2.2) {K};
  \node[v] (M) at (10,-2.2){M};

  % Nó destino
  \node[v, fill=red!12] (T) at (12,0) {T};

  % --- Arestas ajustadas ---
  \draw[->] (S) -- node[above] {30} (A);
  \draw[->] (S) -- node[below] {50} (B);
  \draw[->] (A) -- node[above] {40} (C);
  \draw[->, bend left=25] (E) to node[above left] {10} (S); 
  \draw[->] (E) -- node[left] {10} (A);
  \draw[->] (E) -- node[below] {10} (D);

  \draw[->] (B) -- node[below] {40} (D);
  \draw[->] (D) -- node[below] {30} (I);

  \draw[->, bend left=15] (C) -- node[above] {5}  (I);
  \draw[->] (C) -- node[above] {10} (F);
  
\draw[->, bend left=35] (C) -- node[above right] {5} (H);

  \draw[->, bend right=10] (F) -- node[above] {5} (G);
  \draw[->] (F) -- node[above] {10} (J);

  \draw[->] (G) -- node[above] {10} (H);
  \draw[->, bend left=15] (G) -- node[above] {20} (L);

  \draw[->, bend left=10] (I) -- node[below right] {10} (H);
  \draw[->] (I) -- node[below] {20} (K);

  \draw[->] (H) -- node[above] {40} (K);
  \draw[->] (K) -- node[above] {70} (M);

  \draw[->] (J) -- node[above] {20} (L);
  \draw[->] (L) -- node[above] {40} (T);
  \draw[->, bend right=15] (L) -- node[below right] {30} (H);

  \draw[->, bend left=10] (H) -- node[above right] {20} (T);
  \draw[->] (M) -- node[above] {80} (T);
\end{tikzpicture}






\newpage
\section{Considerações Finais}

O problema de fluxo máximo em redes, juntamente com o método de Ford-Fulkerson, representa uma área de estudo fundamental e de imensa relevância prática na otimização combinatória e na ciência da computação. Como demonstrado ao longo deste artigo, a capacidade de modelar sistemas complexos de transporte, distribuição e comunicação como redes de fluxo permite resolver uma variedade impressionante de problemas do mundo real.

\noindent A metodologia de Ford-Fulkerson, embora concebida há décadas por L.R. Ford Jr. e D.R. Fulkerson, continua sendo a base para a compreensão e o desenvolvimento de algoritmos mais avançados. Seus conceitos centrais de redes residuais, caminhos aumentantes e cortes são indispensáveis para analisar a dinâmica de fluxo e identificar os gargalos (cortes mínimos) que limitam a capacidade de uma rede. 

\noindent A versão Edmonds-Karp, que emprega a busca em largura para encontrar caminhos aumentantes, garante um tempo de execução polinomial, tornando-o uma escolha robusta para muitas aplicações.
No entanto, é crucial reconhecer tanto as vantagens, como a versatilidade e a capacidade de fornecer soluções exatas e identificar vulnerabilidades, quanto as desvantagens, como a potencial ineficiência em sua forma genérica e a sensibilidade a capacidades irracionais. A contínua pesquisa na área, exemplificada pelos algoritmos push-relabel e pelas contribuições de autores como Goldberg e Tarjan, reflete a busca por soluções ainda mais rápidas e eficientes para lidar com redes de grande escala e complexidade.

\noindent A implementação conceitual em Python ilustra como a teoria abstrata pode ser traduzida em código, fornecendo uma ferramenta prática para a análise de redes. A clareza e a flexibilidade do Python facilitam a construção e manipulação das estruturas de dados necessárias, tornando o algoritmo acessível para experimentação e aplicação em diversos contextos.

\noindent Em suma, o problema de fluxo máximo e o algoritmo de Ford-Fulkerson são mais do que apenas um exercício acadêmico; são ferramentas poderosas que capacitam engenheiros, cientistas de dados e pesquisadores a otimizar sistemas, alocar recursos de forma inteligente e tomar decisões estratégicas, contribuindo significativamente para a eficiência e resiliência de infraestruturas modernas. A compreensão de seus princípios e a capacidade de aplicá-los continuam sendo habilidades valiosas no cenário tecnológico atual.




\newpage
\section{Referências Bibliográficas}

\bibliographystyle{sbc}
\bibliography{referencias}

\end{document}
